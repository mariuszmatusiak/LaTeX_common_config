% LaTeX Common Config file
% Copyright (c) 2025 Mariusz Matusiak <mariusz.m.matusiak@gmail.com>
% File: common_cfg.tex

% \documentclass[12pt,a4paper,onecolumn]{article}
% \documentclass{report}
% \documentclass{book}
% \documentclass[letterpaper]{letter}
% \documentclass[landscape]{memoir}

% \usepackage[left=2cm,top=2cm,right=2cm,bottom=2cm]{geometry}  % to modify alignment and margins
\usepackage{amssymb}	            % to enable math symbols (on Linux this should go before babel due to lll issue)
% \usepackage{polski}               % to use Polish encoding on input. Changes "Chapter" to "Rozdział". Or:
\usepackage[polish,english]{babel}  % to use Polish + english encoding (last one is default) on input together, and switch using \selectlanguage{language} with vocab, interpunction and syntax
\usepackage[utf8]{inputenc}         % to configure Unicode text encoding on input as UTF-8
% \usepackage[macce]{inputenc}      % to configure Unicode text encoding on input as UTF-8 on Mac (not working well)
% \usepackage[latin2]{inputenc}     % to configure Unicode text encoding on input as UTF-8 on Linux
% \usepackage[cp1250]{inputenc}     % to configure Unicode text encoding on input as UTF-8 on Windows
\usepackage[T1]{fontenc}            % to configure UTF-8 characters rendering on output
\usepackage{comment}                % to enable comments like \begin{comment}...\end{comment
\usepackage{graphicx}               % to insert images, with rotating and scaling
\usepackage{tabularx}               % to enable tables
\usepackage{multicol}               % to enable multi-column span
\usepackage{multirow}               % to enable multi-row span
\usepackage[unicode,breaklinks,hidelinks]{hyperref}      % to insert bookmarks with polish letters
% \usepackage[hidelinks]{hyperref}  % to hide links frame
\usepackage{amsmath}                % to enable math formulas (incl. polish language), put after lang!
% \usepackage{amssymb}                % to enable math symbols (incl. polish language)
\usepackage{amsfonts}               % to enable math fonts (incl. polish language)
\usepackage{pdfpages}               % to embed other docs/pdf into one pdf
\newcommand\UseBibLaTeX{}           % comment this out if you want to use classic BibTeX instead of BibLaTeX
\ifx\UseBibLaTeX\undefined
\else
    \usepackage[backend=biber,defernumbers=true,sorting=ydnt,style=nature]{biblatex} % to use biber with BibLaTeX instead of default BibTeX
\fi
\usepackage{csquotes}               % a recommended package when babel is used with BibLaTeX
\usepackage{enumitem}               % to customize item list further, not necessary. TODO check if I really need this
% \usepackage[acronym, toc, nogroupskip]{glossaries} % to enable creating glossaries with \makeglossaries and \printglossaries, to be loaded after hyperref
\usepackage[style=indexgroup,abbreviations,postdot,symbols,translate=babel]{glossaries-extra}      % to enable creating glossaries with additional options
\usepackage{makeidx}                % to enable creating index of keywords with \makeindex and \printindex 
\usepackage{IEEEtrantools}          % to enable IEEEeqnarray, TODO check if I really need this
% \usepackage[justification=centering]{caption} % to enable captions. TODO
\usepackage[style=base]{subcaption} % to enable and use subfloat and subfigures.  
% \usepackage[caption=false,justification=centering]{subfig}  % to enable and use subfloat and subfigures. Marked as deprecated. Do the same as above subcaption. Use the above only.
% \usepackage{type1cm}              % to change sizes of AMS fonts. Not needed now.
% \usepackage[titletoc]{appendix}     % package for appendix (titletoc adds \appendixname in toc)
\usepackage{listings}               % to enable code snippets

% A set of additional unicode character declarations
\DeclareUnicodeCharacter{03BB}{$\lambda$}  % λ (U+03BB)
\DeclareUnicodeCharacter{03BC}{$\mu$} % μ (U+03BC)

% Glossaries-extra useful commands:
% - for section, headers and captions - \glsfmtshort{}
% - for text - \gls{}

% Math mode useful commands:
% - normal text in math mode - \textrm{}
% - italic text in math mode - \mathit{}
